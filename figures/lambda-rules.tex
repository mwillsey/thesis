\begin{figure}
\begin{subfigure}[t]{0.48\linewidth}
  \begin{lstlisting}[language=Rust, basicstyle=\tiny\ttfamily, numbers=left, escapechar=|,
   xleftmargin=13pt,
   numbersep=7pt,
]
define_language! {
  enum Lambda {
    // enum variants have data or children (eclass Ids)
    // [Id; N] is an array of N `Id`s

    // base type operators
    "+" = Add([Id; 2]), "=" = Eq([Id; 2]),
    "if" = If([Id; 3]),

    // functions and binding
    "app" = App([Id; 2]), "lam" = Lambda([Id; 2]),
    "let" = Let([Id; 3]), "fix" = Fix([Id; 2]),

    // (var x) is a use of `x` as an expression
    "var" = Use(Id),
    // (subst a x b) substitutes a for (var x) in b
    "subst" = Subst([Id; 3]),

    // base types have no children, only data
    Bool(bool), Num(i32), Symbol(String),
  }
}

// example terms and what they simplify to
// pulled directly from the |\egg|test suite

test_fn! { lambda_under, rules(),
  "(lam x (+ 4 (app (lam y (var y)) 4)))"
  => "(lam x 8))",
}

test_fn! { lambda_compose_many, rules(),
  "(let compose (lam f (lam g (lam x
                (app (var f)
                     (app (var g) (var x))))))
   (let add1 (lam y (+ (var y) 1))
   (app (app (var compose) (var add1))
        (app (app (var compose) (var add1))
             (app (app (var compose) (var add1))
                  (app (app (var compose) (var add1))
                       (var add1)))))))"
  => "(lam ?x (+ (var ?x) 5))"
}

test_fn! { lambda_if_elim, rules(),
  "(if (= (var a) (var b))
       (+ (var a) (var a))
       (+ (var a) (var b)))"
  => "(+ (var a) (var b))"
}\end{lstlisting}
\end{subfigure}
\hfill
\begin{subfigure}[t]{0.48\linewidth}
  \begin{lstlisting}[language=Rust, basicstyle=\tiny\ttfamily, escapechar=|, numbers=left, firstnumber=51,
  numbersep=7pt,
]
// Returns a list of rewrite rules
fn rules() -> Vec<Rewrite<Lambda, LambdaAnalysis>> { vec![

 // open term rules
 rw!("if-true";  "(if  true ?then ?else)" => "?then"),
 rw!("if-false"; "(if false ?then ?else)" => "?else"),
 rw!("if-elim";  "(if (= (var ?x) ?e) ?then ?else)" => "?else"
     if ConditionEqual::parse("(let ?x ?e ?then)",
                              "(let ?x ?e ?else)")),
 rw!("add-comm";  "(+ ?a ?b)"        => "(+ ?b ?a)"),
 rw!("add-assoc"; "(+ (+ ?a ?b) ?c)" => "(+ ?a (+ ?b ?c))"),
 rw!("eq-comm";   "(= ?a ?b)"        => "(= ?b ?a)"),

 // substitution introduction
 rw!("fix";     "(fix ?v ?e)" =>
                "(let ?v (fix ?v ?e) ?e)"),
 rw!("beta";    "(app (lam ?v ?body) ?e)" =>
                "(let ?v ?e ?body)"),

 // substitution propagation
 rw!("let-app"; "(let ?v ?e (app ?a ?b))" =>
                "(app (let ?v ?e ?a) (let ?v ?e ?b))"),
 rw!("let-add"; "(let ?v ?e (+   ?a ?b))" =>
                "(+   (let ?v ?e ?a) (let ?v ?e ?b))"),
 rw!("let-eq";  "(let ?v ?e (=   ?a ?b))" =>
                "(=   (let ?v ?e ?a) (let ?v ?e ?b))"),
 rw!("let-if";  "(let ?v ?e (if ?cond ?then ?else))" =>
                "(if (let ?v ?e ?cond)
                     (let ?v ?e ?then)
                     (let ?v ?e ?else))"),

 // substitution elimination
 rw!("let-const";    "(let ?v ?e ?c)" => "?c"
     if is_const(var("?c"))),
 rw!("let-var-same"; "(let ?v1 ?e (var ?v1))" => "?e"),
 rw!("let-var-diff"; "(let ?v1 ?e (var ?v2))" => "(var ?v2)"
     if is_not_same_var(var("?v1"), var("?v2"))),
 rw!("let-lam-same"; "(let ?v1 ?e (lam ?v1 ?body))" =>
                     "(lam ?v1 ?body)"),
 rw!("let-lam-diff"; "(let ?v1 ?e (lam ?v2 ?body))" =>
     ( CaptureAvoid {
        fresh: var("?fresh"), v2: var("?v2"), e: var("?e"),
        if_not_free: "(lam ?v2 (let ?v1 ?e ?body))"
                     .parse().unwrap(),
        if_free: "(lam ?fresh (let ?v1 ?e
                              (let ?v2 (var ?fresh) ?body)))"
                 .parse().unwrap(),
     })
     if is_not_same_var(var("?v1"), var("?v2"))),
]}\end{lstlisting}
\end{subfigure}
\vspace{-0.5em}
\caption[Language and rewrites for the lambda calculus in \egg]{
\egg is generic over user-defined languages;
  here we define a language and rewrite rules for a lambda calculus partial evaluator.
The provided \texttt{define\_language!} macro (lines 1-22) allows the simple definition
  of a language as a Rust \texttt{enum}, automatically deriving parsing and
  pretty printing.
A value of type \texttt{Lambda} is an \enode that holds either data that the
  user can inspect or some number of \eclass children (\eclass \texttt{Id}s).

Rewrite rules can also be defined succinctly (lines 51-100).
Patterns are parsed as s-expressions:
  strings from the \texttt{define\_language!} invocation (ex: \texttt{fix}, \texttt{=}, \texttt{+}) and
  data from the variants (ex: \texttt{false}, \texttt{1}) parse as operators or terms;
  names prefixed by ``\texttt{?}'' parse as pattern variables.

Some of the rewrites made are conditional using the
  ``\texttt{left => right if cond}''
  syntax.
The \texttt{if-elim} rewrite on line 57 uses \egg's provided
  \texttt{ConditionEqual} as a condition, only applying the right-hand side
  if the \egraph can prove the two argument patterns equivalent.
The final rewrite, \texttt{let-lam-diff}, is dynamic to support capture avoidance;
  the right-hand side is a Rust value that
  implements the \texttt{Applier} trait instead of a pattern.
\autoref{fig:lambda-analysis} contains the supporting code for these rewrites.

We also show some of the tests (lines 27-50)
  from \egg's \texttt{lambda} test suite.
The tests proceed by inserting the term on the left-hand side, running
  \egg's equality saturation, and then checking to make sure the right-hand
  pattern can be found in the same \eclass as the initial term.
}
\label{fig:lambda-rules}
\label{fig:lambda-lang}
\label{fig:lambda-examples}
\end{figure}

%%% Local Variables:
%%% TeX-master: "../thesis"
%%% End:
